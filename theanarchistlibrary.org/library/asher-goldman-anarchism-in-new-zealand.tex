\documentclass[DIV=12,%
               BCOR=10mm,%
               headinclude=false,%
               footinclude=false,open=any,%
               fontsize=11pt,%
               twoside,%
               paper=210mm:11in]%
               {scrbook}






\usepackage[noautomatic]{imakeidx}



\usepackage{microtype}
\usepackage{graphicx}
\usepackage{alltt}
\usepackage{verbatim}
\usepackage[shortlabels]{enumitem}
\usepackage{tabularx}
\usepackage[normalem]{ulem}
\def\hsout{\bgroup \ULdepth=-.55ex \ULset}
% https://tex.stackexchange.com/questions/22410/strikethrough-in-section-title
% Unclear if \protect \hsout is needed. Doesn't looks so
\DeclareRobustCommand{\sout}[1]{\texorpdfstring{\hsout{#1}}{#1}}
\usepackage{wrapfig}

% avoid breakage on multiple <br><br> and avoid the next [] to be eaten
\newcommand*{\forcelinebreak}{\strut\\*{}}

\newcommand*{\hairline}{%
  \bigskip%
  \noindent \hrulefill%
  \bigskip%
}

% reverse indentation for biblio and play

\newenvironment*{amusebiblio}{
  \leftskip=\parindent
  \parindent=-\parindent
  \smallskip
  \indent
}{\smallskip}

\newenvironment*{amuseplay}{
  \leftskip=\parindent
  \parindent=-\parindent
  \smallskip
  \indent
}{\smallskip}

\newcommand*{\Slash}{\slash\hspace{0pt}}


% http://tex.stackexchange.com/questions/3033/forcing-linebreaks-in-url
\PassOptionsToPackage{hyphens}{url}\usepackage[hyperfootnotes=false,hidelinks,breaklinks=true]{hyperref}
\usepackage{bookmark}

\usepackage{fontspec}
\usepackage{polyglossia}
\setmainlanguage{english}
\setmainfont{LinLibertine_R.otf}[Script=Latin,%
 Ligatures=TeX,%
 Path=/usr/share/fonts/opentype/linux-libertine/,%
 BoldFont=LinLibertine_RB.otf,%
 BoldItalicFont=LinLibertine_RBI.otf,%
 ItalicFont=LinLibertine_RI.otf]
\setmonofont{cmuntt.ttf}[Script=Latin,%
 Ligatures=TeX,%
 Scale=MatchLowercase,%
 Path=/usr/share/fonts/truetype/cmu/,%
 BoldFont=cmuntb.ttf,%
 BoldItalicFont=cmuntx.ttf,%
 ItalicFont=cmunit.ttf]
\setsansfont{cmunss.ttf}[Script=Latin,%
 Ligatures=TeX,%
 Scale=MatchLowercase,%
 Path=/usr/share/fonts/truetype/cmu/,%
 BoldFont=cmunsx.ttf,%
 BoldItalicFont=cmunso.ttf,%
 ItalicFont=cmunsi.ttf]
\newfontfamily\englishfont{LinLibertine_R.otf}[Script=Latin,%
 Ligatures=TeX,%
 Path=/usr/share/fonts/opentype/linux-libertine/,%
 BoldFont=LinLibertine_RB.otf,%
 BoldItalicFont=LinLibertine_RBI.otf,%
 ItalicFont=LinLibertine_RI.otf]



\renewcommand*{\partpagestyle}{empty}


% global style

\pagestyle{plain}




\usepackage{indentfirst}

% remove the numbering
\setcounter{secnumdepth}{-2}

% remove labels from the captions
\renewcommand*{\captionformat}{}
\renewcommand*{\figureformat}{}
\renewcommand*{\tableformat}{}
\KOMAoption{captions}{belowfigure,nooneline}
\addtokomafont{caption}{\centering}







\deffootnote[3em]{0em}{4em}{\textsuperscript{\thefootnotemark}~}


\addtokomafont{disposition}{\rmfamily}
\addtokomafont{descriptionlabel}{\rmfamily}


\frenchspacing
% avoid vertical glue
\raggedbottom

% this will generate overfull boxes, so we need to set a tolerance
% \pretolerance=1000
% pretolerance is what is accepted for a paragraph without
% hyphenation, so it makes sense to be strict here and let the user
% accept tweak the tolerance instead.
\tolerance=200
% Additional tolerance for bad paragraphs only
\setlength{\emergencystretch}{30pt}

% (try to) forbid widows/orphans
\clubpenalty=10000
\widowpenalty=10000





% given that we said footinclude=false, this should be safe
\setlength{\footskip}{2\baselineskip}

\title{Anarchism in New Zealand}
\date{2009}
\author{Asher Goldman}
\subtitle{}




% https://groups.google.com/d/topic/comp.text.tex/6fYmcVMbSbQ/discussion
\hypersetup{%
pdfencoding=auto,
pdftitle={Anarchism in New Zealand},%
pdfauthor={Asher Goldman},%
pdfsubject={},%
pdfkeywords={New Zealand; history}%
}


\begin{document}






  \begin{titlepage}


  \strut\vskip 2em
  \begin{center}

  {\usekomafont{title}{\huge Anarchism in New Zealand\par}}%
  \vskip 1em
  
  \vskip 2em
  
  {\usekomafont{author}{Asher Goldman\par}}%
  
  \vskip 1.5em



   
   \vfill



  
  {\usekomafont{date}{2009\par}}%
  


  \end{center}


     \end{titlepage}
     
     \cleardoublepage



\tableofcontents
% start a new right-handed page

\cleardoublepage




As in much of the world, anarchists and sympathizers played a large role in the formation of syndicalist-leaning unions in New Zealand in the early 1900s. World War I, the 1917 Russian Revolution, and the formation of the Labor Party combined to almost completely decimate the movement until the late 1950s, when it reformed as part of the New Left. Later, the influence of punk and then the anti-globalization movement would help grow anarchism’s popularity, although it has still not begun to approach the level of influence it held in the early twentieth century.


Perhaps the first to call themselves anarchists in New Zealand were several activists in the New Zealand Socialist Party, formed in 1901. Within the party, Wellington became the center for a group of anti-parliamentary socialists. In 1908 a 3,000 member Socialist Party held a conference at which parliamentary action was condemned by a two to one majority. In 1913 anarchists in Wellington formed the Freedom Group, which was New Zealand’s first recorded explicitly anarchist grouping. First formed in Chicago, USA, in 1905, the Industrial Workers of the World (IWW) had branches in several centers across New Zealand, starting with Wellington in 1908. Membership included many anarchists. This revolutionary union had significant impact on the 1912 strike in the gold mining town of Waihi.


Anarchism would not rear its head again in New Zealand until the late 1950s, when anarchists became involved in protest movements and socialist discussion groups. By the late 1960s the new left Progressive Youth Movement had branches in many centers. The Wellington and Christchurch branches especially had a high number of anarchists involved. The early 1970s saw Resistance bookshops formed in the three main cities as centers for activism, the longest of which lasted only seven years. Punk anarchists began to get active in the early 1980s, and in one of New Zealand’s few political bombings, a punk blew himself up (possibly purposefully) while attempting to attack the Wanganui Police computer.


In 1987 Auckland anarchists founded \emph{State Adversary} magazine, which covered anarchist news and views from around the country until 2000. The year 1995 saw the opening of the Freedom Shop, a Wellington anarchist infoshop, the longest running in New Zealand history (it remains open in 2008). In the wake of anti-globalization protests in Seattle, Prague, and other cities, anarchism in New Zealand welcomed an influx of new adherents, and regular protest activity increased for a period in the main centers. In October 2007 anarchists were among 16 arrested in the first nationwide anti-terrorism raids, although terror charges were ultimately not laid, with the 16 facing firearms charges instead.


SEE ALSO: Anarchism ; Anarchism, Australia ; Communist Party NZ and the New Zealand Revolutionary Left ; Industrial Workers of the World (IWW)

\section{References And Suggested Readings}



\begin{amusebiblio}


Boraman, T. (2002) The New Left in New Zealand. In P. Moloney \& K. Taylor (Eds.), \emph{On the Left: Essays on Socialism in New Zealand}. Dunedin: Otago University Press.


Boraman, T. (2007) \emph{Rabble Rousers and Merry Pranksters: A History of Anarchism in Aotearoa\Slash{}New Zealand from the Mid-1950s to the Early 1980s}. Christchurch: Katipo Books and Irrecuperable Press.


Buchanan, S. (1999) \emph{Anarchy: The Transmogrification of Everyday Life}. Wellington: Committee for the Establishment of Civilization.


Prebble, F. (1995) \emph{Trouble Makers: Anarchism and Syndicalism, the Early Years of the Libertarian Movement in Aotearoa\Slash{}New Zealand}. Christchurch: Libertarian Press.


Steiner, P. (2007) The History of the IWW in New Zealand. In P. Steiner \& F. Hanlon, \emph{Industrial Unionism}. Wellington: Rebel Press.



\end{amusebiblio}







% begin final page

\clearpage


% if we are on an odd page, add another one, otherwise when imposing
% the page would be odd on an even one.
\ifthispageodd{\strut\thispagestyle{empty}\clearpage}{}


% new page for the colophon

\thispagestyle{empty}

\begin{center}

The Anarchist Library



\smallskip
Anti-Copyright



\bigskip
\includegraphics[width=0.25\textwidth]{logo-en}
\bigskip

\end{center}

\strut

\vfill

\begin{center}


Asher Goldman

Anarchism in New Zealand



2009


\bigskip


Goldman, Asher. “Anarchism, New Zealand.” In \emph{The International Encyclopedia of Revolution and Protest}: \emph{1500 to the Present}, edited by Immanuel Ness, 139–140. Vol. 1. Malden, MA: Wiley-Blackwell, 2009.
   

       

       

       

       

       
   



\bigskip
\textbf{theanarchistlibrary.org}


\end{center}

% end final page with colophon


\end{document}


% No format ID passed.



